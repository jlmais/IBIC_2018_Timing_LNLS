\documentclass[a4paper,
               %boxit,        % check whether paper is inside correct margins
               %titlepage,    % separate title page
               %refpage       % separate references
               biblatex,      % biblatex is used
               %keeplastbox,  % flushend option: not to un-indent last line in References
               %nospread,     % flushend option: do not fill with whitespace to balance columns
               %hyphens,      % allow \url to hyphenate at "-" (hyphens)
               %xetex,        % use XeLaTeX to process the file
               %luatex,       % use LuaLaTeX to process the file
               ]{jacow}
%
% ONLY FOR \footnote in table/tabular
%
\usepackage{pdfpages,multirow,ragged2e} %
%
% CHANGE SEQUENCE OF GRAPHICS EXTENSION TO BE EMBEDDED
% ----------------------------------------------------
% test for XeTeX where the sequence is by default eps-> pdf, jpg, png, pdf, ...
%    and the JACoW template provides JACpic2v3.eps and JACpic2v3.jpg which
%    might generates errors, therefore PNG and JPG first
%
\makeatletter%
	\ifboolexpr{bool{xetex}}
	 {\renewcommand{\Gin@extensions}{.pdf,%
	                    .png,.jpg,.bmp,.pict,.tif,.psd,.mac,.sga,.tga,.gif,%
	                    .eps,.ps,%
	                    }}{}
\makeatother

% CHECK FOR XeTeX/LuaTeX BEFORE DEFINING AN INPUT ENCODING
% --------------------------------------------------------
%   utf8  is default for XeTeX/LuaTeX
%   utf8  in LaTeX only realises a small portion of codes
%
\ifboolexpr{bool{xetex} or bool{luatex}} % test for XeTeX/LuaTeX
 {}                                      % input encoding is utf8 by default
 {\usepackage[utf8]{inputenc}}           % switch to utf8

\usepackage[USenglish]{babel}

\addbibresource{citations.bib}
\bibliography{citations}

% Don't justify URLs in the references
\renewcommand*{\bibfont}{\raggedright}

%%
%%   Lengths for the spaces in the title
%%   \setlength\titleblockstartskip{..}  %before title, default 3pt
%%   \setlength\titleblockmiddleskip{..} %between title + author, default 1em
%%   \setlength\titleblockendskip{..}    %afterauthor, default 1em

\begin{document}

\title{A MicroTCA.4  Timing Receiver for the Sirius Timing System}

\author{J.L.N. Brito\thanks{joao.brito@lnls.br}, S.R. Marques, D.O. Tavares, L.M. Russo, G.B.M. Bruno, LNLS, Campinas, Brazil}
	
\maketitle

%
\begin{abstract}
   The AMC FMC carrier (AFC) is a MicroTCA.4 AMC board which has a very flexible clock circuit that enables any clock source to be connected to any clock input, including telecom clock, FMC clocks, programmable VCXO oscillator and FPGA. This paper presents the use of the AFC board as an event receiver connected to the Sirius timing system to provide low jitter synchronized clocks and triggers for Sirius BPM electronics and other devices.
\end{abstract}


\section{introduction}
Sirius timing system is an event based, as described in \cite{timing_icalepcs15}, where an EVG broadcasts a data frame ... the event receivers decode the data to output triggers and clock synchronized to RF for diagnostics and injection devices as well as BPM electronics.   
The Sirius BPM electronics is composed by an analog standalone module, an ADC and a DSP board ... MicroTCA.4 platform was adopted ... The AMC FMC carrier (AFC)~\cite{afc-git} is a MicroTCA.4 AMC board partially based on Simple PCIe FMC carrier (SPEC)~ \cite{spec} design. 
The AFC board was designed to be an open-source platform for Sirius BPM and orbit feedback systems \cite{ebpm_icalepcs13}.

Thanks to the flexible clock circuits and triggers distribution options available in AFC board and MicroTCA.4 platform we decide to develop and timing receiver board based on the same hardware platform... 

\section{hardware}
Short review about hardware focusing on timing application.

\subsection{AFC}

It has 8 bidirectional trigger lines in the AMC connector... Flexible clock circuits... HPC FMC connectors with n differential clocks and n triggers... Artix-7 FPGA ...

\begin{figure}[!htb]
   \centering
   \includegraphics*[width=0.8\columnwidth]{AFC_POFs_resized}
   \caption{AFC mounted with 2 FMC-POF.}
   \label{fig:afc_pofs}
\end{figure}

\subsection{FMC-POF}
The FMC-POF \cite{fmc-pof-git} was designed to provide 5 POF outputs to triggers external devices near to the BPM Crate such as current power supplies.

\begin{figure}[!htb]
   \centering
   \includegraphics*[width=0.5\columnwidth]{FMC_POF_resized}
   \caption{FMC-POF.}
   \label{fig:fmc_pof}
\end{figure}

\subsection{RTM-SFP}
The RTM-SFP \cite{rtm-sfp-git} is a Rear Transition Module board MicroTCA.4 standard. Its main components are 8 SFP connectors to provide optical fiber interface with timing system and a Si570 programmable oscillator to output a reference clock to GTP transceivers. 

\begin{figure}[!htb]
   \centering
   \includegraphics*[width=0.8\columnwidth]{RTM_SFP_resized}
   \caption{RTM with 8 SFP.}
   \label{fig:rtm_sfp}
\end{figure}

\section{FPFA Gateware}

This section describes the main aspects of the FPGA gateware to implement an event receiver and a frequency and phase locked loop to output a low jitter synchronized clock which is the reference to the ADCs of the BPM electronics.

\subsection{Event Receiver}

Describes configurable parameters implemented to the event receiver like pulse width, delay, polarity and so on. Take the EVE and EVR description in the last paper as an example.

\subsection{Low Jitter Synchronized Clock}
The main purpose of this project is providing reference clock to the ADCs of the BPM electronics. The defined frequency to be the reference was $\frac{5}{36}RF$
The PLL FPGA block generates from event clock the reference clock and  
\[f_{dmtd} = f_{ref}\frac{N}{N+1}\]

\begin{figure}[!htb]
   \centering
   \includegraphics*[width=0.8\columnwidth]{AFCRefClockLoop}
   \caption{Layout of frequency and phase loop.}
   \label{fig:AFCRefClockLoop}
\end{figure}

\begin{figure}[!htb]
   \centering
   \includegraphics*[width=0.8\columnwidth]{AFCFPGADMTD}
   \caption{FPGA frequency and phase controller.}
   \label{fig:AFCFPGADMTD}
\end{figure}

\section{Software Interface}
Describe the software interface supported HALCS and EPICS IOC.

\section{CONCLUSION}

Analyse tests?

\printbibliography
\newpage

\end{document}
